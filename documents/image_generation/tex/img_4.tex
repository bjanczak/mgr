\documentclass{article}
\usepackage{tikz,amsmath,siunitx}
\usepackage{color}
\usetikzlibrary{arrows,snakes,backgrounds,patterns,matrix,shapes,fit,calc,shadows,plotmarks}
\usepackage[graphics,tightpage,active]{preview}
\PreviewEnvironment{tikzpicture}
\PreviewEnvironment{equation}
\PreviewEnvironment{equation*}
\newlength{\imagewidth}
\newlength{\imagescale}
\pagestyle{empty}
\thispagestyle{empty}
\begin{document}
%%%%%
\begin{tikzpicture}[scale=0.4]
\filldraw [black]
(1,4)     circle (3pt) 
(2,2)     circle (3pt)
(2.5,4)   circle (3pt) 
(3,3)     circle (3pt)
(4,4)     circle (3pt) 
(3,5)     circle (3pt)
(3.5,2)   circle (3pt) 
(4.5,2)   circle (3pt)
(5,1)     circle (3pt)  
(6,1.5)   circle (3pt) 
(5.5,2.5) circle (3pt) 
(5,4)     circle (3pt)
(6,3.5)   circle (3pt) 
(7,2)     circle (3pt)
(7,3)     circle (3pt)
(7,4)     circle (3pt)
(8,3)     circle (3pt)
(8,5)     circle (3pt)
(8.5,4)   circle (3pt)
(4,1)     circle (3pt)
(5.5,6)   circle (3pt);
%okręgi zbioru drugiego
\draw[color=black]
(5.5,6)   circle (2.2) node[anchor=south] {$p$}
(5,4)     circle (2.2)
(6,3.5)   circle (2.2) node[anchor=west] {$q$};
\draw[->, very thick] (6,3.5) -- (5,4);
\draw[->, very thick] (5,4) -- (5.5,6);
\end{tikzpicture}
%%%%%
\end{document}